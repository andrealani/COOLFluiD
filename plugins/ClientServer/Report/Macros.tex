\newcommand{\cf}{COOLFluiD } 

\oddsidemargin -4mm % Decreases the left margin by 4mm
\textwidth 17cm %Sets text width across page = 17cm
\textheight 22cm %Sets text height up and down = 20cm

\newcommand{\dateHistory}[1]{
\noindent\textit{\textbf{\underline{#1}}}}

\newcommand{\seeSingle}[1]{(see section \ref{#1}, page \pageref{#1})}
\newcommand{\seeChapter}[1]{(see chapter \ref{#1}, page \pageref{#1})}
\newcommand{\seeAlso}[2]{(see also \texttt{#2}, section \ref{#1}, page
\pageref{#1})}
\newcommand{\see}[2]{\seeSingle{#1_#2}}
\newcommand{\createLabel}[2]{\label{#1_#2}}
\newcommand{\inQuotes}[1]{\textquotedblleft#1\textquotedblright}

\setcounter{secnumdepth}{5}

\renewcommand{\thesection}{\thechapter.\arabic{section}}
\renewcommand{\thesubsection}{\thesection.\arabic{subsection}}
\renewcommand{\thesubsubsection}{\thesubsection.\arabic{subsubsection}}
\renewcommand{\theparagraph}{\thesubsubsection.\arabic{paragraph}}

\renewcommand{\texttt}[1]{\lstinline{#1}}

\let\sectionv\section
\renewcommand{\section}[1]{\sectionv{#1} \setcounter{subsection}{0} 
\setcounter{paragraph}{0}}

\let\subsectionv\subsection
\renewcommand{\subsection}[1]{\subsectionv{#1} \setcounter{paragraph}{0} 
\setcounter{subsubsection}{0}}

\newcommand{\descrXMLAttr}[2]{\indent \ \ \texttt{#1} : #2}

\lstset{language=QtCpp,%
	commentstyle=\scriptsize\ttfamily\slshape,
	basicstyle=\scriptsize\ttfamily,
	keywordstyle=[0]{\scriptsize\rmfamily\textit},
	keywordstyle=[1]{\scriptsize\rmfamily\bfseries},
	keywordstyle=[2]{\scriptsize\rmfamily\bfseries},
	keywordstyle=[3]{\scriptsize\rmfamily\bfseries},
	keywordstyle=[4]{\scriptsize\rmfamily\bfseries},
	keywordstyle=[5]{\scriptsize\rmfamily},
	backgroundcolor=\color[rgb]{.95,.95,.95},
	framerule=0.5pt,%
	frame=trbl,%
	tabsize=1}

\makeatletter
% Une commande sembleble à \rlap ou \llap, mais centrant son argument
\def\clap#1{\hbox to 0pt{\hss #1\hss}}%
% Une commande centrant son contenu (à utiliser en mode vertical)
\def\ligne#1{%
  \hbox to \hsize{%
    \vbox{\centering #1}}}%
% Une comande qui met son premier argument à gauche, le second au
% milieu et le dernier à droite, la première ligne ce chacune de ces
% trois boites coïncidant
\def\haut#1#2#3{%
  \hbox to \hsize{%
    \rlap{\vtop{\raggedright #1}}%
    \hss
    \clap{\vtop{\centering #2}}%
    \hss
    \llap{\vtop{\raggedleft #3}}}}%
% Idem, mais cette fois-ci, c'est la dernière ligne
\def\bas#1#2#3{%
  \hbox to \hsize{%
    \rlap{\vbox{\raggedright #1}}%
    \hss
    \clap{\vbox{\centering #2}}%
    \hss
    \llap{\vbox{\raggedleft #3}}}}%
% La commande \maketitle
\def\maketitle{%
  \thispagestyle{empty}\vbox to \vsize{%
    \haut{}{\@blurb}{}
    \vfill
    \ligne{\Huge \textit{\@title}}
    \vspace{1cm}
    \ligne{\Large \@author}
    \vspace{1cm}
%     \ligne{<gquentin@gmail.com>}
    \vfill
    \vfill
    \bas{}{Stage master : Professor Herman \textsc{Deconinck}\\
    Advisor : Doctor Tiago \textsc{Quintino}\\
    Reference teacher : Nicolas \textsc{Vansteenkiste}\\[1em]
    June 2009}{}
    }%
  \cleardoublepage
  }

% Les commandes permettant de définir la date, le lieu, etc.
\def\date#1{\def\@date{#1}}
\def\author#1{\def\@author{#1}}
\def\title#1{\def\@title{#1}}
\def\blurb#1{\def\@blurb{#1}}

\makeatother
  \author{Quentin \textsc{Gasper}}
  \date{June 2009}
  \blurb{%
    Haute \'{E}cole de Bruxelles \\
    \'{E}cole Sup\'{e}rieure d'Informatique \\[1em]
    Report for stage done at \\ Von Karman Institute for Fluid Dynamics \\
    }%
