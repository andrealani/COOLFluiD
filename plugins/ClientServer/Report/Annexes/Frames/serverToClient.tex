\chapter{From server to client}

\section{Tree}
\label{tree}

\subsection{Description}

This frame is used by the server to send the tree.

\subsection{Structure}

\begin{lstlisting}[language=XML]
<ClientServerXML>
  <tree>
    (tree)
  </tree>
</ClientServerXML>
\end{lstlisting}

\subsection{Data structure}

Data represent the tree as returned by the simulator.

%-------------------------------------------------------------
%-------------------------------------------------------------

\section{Message}
\label{message}

\subsection{Description}

This frame is used by the server to send a message to the client.

\subsection{Structure}

\begin{lstlisting}[language=XML]
<ClientServerXML>
  <message value="Message text" />
</ClientServerXML>
\end{lstlisting}

\subsection{Arguments}
\descrXMLAttr{value}{Message text}

%-------------------------------------------------------------
%-------------------------------------------------------------

\section{Error}
\label{error}

\subsection{Description}

This frame is used by the server to send a message describing an error to
the client.

\subsection{Structure}

\begin{lstlisting}[language=XML]
<ClientServerXML>
  <error value="Error message" />
</ClientServerXML>
\end{lstlisting}

\subsection{Arguments}
\descrXMLAttr{value}{Error message text}

%-------------------------------------------------------------
%-------------------------------------------------------------

\section{Abstract types}
\label{abstractTypes}

\subsection{Description}

This frame is used by the server to send abstract types list to the client. 

\subsection{Structure}

\begin{lstlisting}[language=XML]
<ClientServerXML>
  <abstractTypes typeName="typeName" typesList="type1, type2, ..." />
</ClientServerXML>
\end{lstlisting}

\subsection{Arguments}
\descrXMLAttr{typeName}{Name of the type containing these abstract types.\\}
\descrXMLAttr{typesList}{Types list. Items are separated by the string ", ".}

%-------------------------------------------------------------
%-------------------------------------------------------------

\section{Concrete types}
\label{concreteTypes}

\subsection{Description}

This frame is used by the server to send concrete types list to the client.

\subsection{Structure}

\begin{lstlisting}[language=XML]
<ClientServerXML>
  <concreteTypes typeName="typeName" typesList="type1, type2, ..." />
</ClientServerXML>
\end{lstlisting}

\subsection{Arguments}
\descrXMLAttr{typeName}{Name of the abstract type containing these concrete
types.\\}
\descrXMLAttr{typesList}{Types list. Items are separated by the string ", ".}

%-------------------------------------------------------------
%-------------------------------------------------------------

\section{Directory contents}
\label{dirContent}

\subsection{Description}

This frame is used by the server to a directory contents.

\subsection{Structure}

\begin{lstlisting}[language=XML]
<ClientServerXML>
  <dirContent path="/path/to/dir" dirs="dir1*dir2*dir3..." 
      files="file1*file2*file3..." />
</ClientServerXML>
\end{lstlisting}

\subsection{Arguments}
\descrXMLAttr{path}{Directory path.\\}
\descrXMLAttr{dir}{Sub-directories list. Items are separated by an asterisk.\\}
\descrXMLAttr{files}{Files list. Items are separated by an asterisk.}

%-------------------------------------------------------------
%-------------------------------------------------------------

\section{Acknowledgement}
\label{ack}

\subsection{Description}

This frame is used by the server to indicate that the specified command
(specified by the type of its frame) succeeded.

\subsection{Structure}

\begin{lstlisting}[language=XML]
<ClientServerXML>
  <ack type="typeToAck" />
</ClientServerXML>
\end{lstlisting}

\subsection{Arguments}
\descrXMLAttr{type}{Type to acknowledge.\\}

%-------------------------------------------------------------
%-------------------------------------------------------------

\section{Non-acknowledgement}
\label{nack}

\subsection{Description}

This frame is used by the server to indicate that the specified command
(specified by the type of its frame) failed.

\subsection{Structure}

\begin{lstlisting}[language=XML]
<ClientServerXML>
  <nack type="typeToNack" />
</ClientServerXML>
\end{lstlisting}

\subsection{Arguments}
\descrXMLAttr{type}{Type to non-acknowledge.\\}

%-------------------------------------------------------------
%-------------------------------------------------------------

\section{Simulation running}
\label{simulationRunning}

\subsection{Description}

This frame is used by the server to indicate that the simulation is running.
Used to notify a new client that a simulation is already running.

\subsection{Structure}

\begin{lstlisting}[language=XML]
<ClientServerXML>
  <simulationRunning />
</ClientServerXML>
\end{lstlisting}