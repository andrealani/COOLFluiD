\chapter{Introduction}

\section{Stage objectives}

Until now, a simulation configuration was written by hand in a file which was
then used in batch mode. This fils has many parameters, which are the source of
many errors. My project was to provide a graphical user interface (GUI) that
allows user to configure more easily a simulation and run
it. 

\section{COOLFluiD}

\inQuotes{\textit{\cf}} is a simulation environment developed at the Von Karman
Institute for Fluid Dynamics. This environment solves complex physical problems
using high-performant algorithms. This is the project to which I contributed
during my stage.\\

\section{Von Karman Institute for Fluid Dynamics}

Von Karman Institute for Fluid Dynamics is an international center in
Rhode-St-Gen\`{e}se for scientific research and advanced scientific learning. It
hosts many European PhD candidates and stagiaires.\\

The Institute was founded by Th\'{e}odore von Karman in 1956 and has sponsors
like governmental and international agencies as well as industries.
\\

The Institute has three departments :
\begin{itemize}
 \item \textbf{aeronautics and aerospace} : focuses on the modeling, simulation
and experimental validation of atmospheric entry flows and thermal protection
systems (TPS), including transition to turbulence and stability. \cf is
developped in this department, therefore this is where I worked.

 \item \textbf{turbomachinery and propulsion} : specializes in aero-thermal
aspects of turbomachinery components for aero-engines and industrial gas
turbines. The department has acquired a world recognised expertise on
steady/unsteady aerodynamic and aero/thermal aspects of high pressure, including
cooling, and low pressure turbomachinery components through the design,
development and use of a number of unique wind tunnels. The department has over
20 years of experience in the analysis of flow in turbomachines, and in the
design techniques and multi-disciplinary optimization methods or their
components.

 \item \textbf{environmental and applied fluid dynamics} : covers all kinds of
activities complementary to the other two departments related to fluid dynamics
in the academic and industrial world. It has a large expertise in the study of
aeroacoustics, multiphase flows, vehicle aerodynamics, biological  flows and
environmental flows (including the study of interaction between atmospheric
winds and human activities).
\end{itemize}

Two publicly known projects where this Institute is involved are the Princess
Elisabeth Station (November 2007) and the study of the fastest bicycle in the
world (March 2009).
